\documentclass[xcolor=dvipsnames,notes]{beamer}
\usecolortheme[named=Brown]{structure}
\usetheme{default}
\setbeamertemplate{navigation symbols}{} 
\usepackage{tikz}
\usetikzlibrary{arrows,decorations.pathmorphing,backgrounds,positioning,fit}
\usetikzlibrary{datavisualization.formats.functions}
\usetikzlibrary{shapes}
\usetikzlibrary{calc,patterns,angles,quotes}
\include{macro}
\usepackage{epsfig}
\usepackage{natbib}
\usepackage{graphicx}
\usepackage{multimedia}
\usepackage{verbatim}
\include{acmmacro}
\begin{document}
%\setbeamercolor{titlelike}{fg=gray,bg=white}
%\setbeamercolor{itemize item}{fg=gray,bg=white}
%\setbeamercolor{enumerate item}{fg=gray,bg=white}
%\setbeamercolor{block title}{fg=black,bg=white}
%==============================================
\title{TPG4190 Seismic data acquisition and processing \\
               Lecture 17: Multiples - Radon demultiple}
\author{B. Arntsen}
\institute[NTNU]{
  NTNU\\
  Department of Geoscience and petroleum \\
  \texttt{borge.arntsen@ntnu.no}
}
\date{Trondheim fall 2020}
\begin{frame}
 \titlepage
\end{frame}
%
%==============================================
\begin{frame}{Overview}
%==============================================
\begin{itemize}
  \item The effect of the free surface
  \item The Radon transform
  \item Radon demultiple
\end{itemize}
\end{frame}
%
%==============================================
\begin{frame}{The Radon transform}
%==============================================
\begin{itemize}
   \item 2D Radon transform of a function
\end{itemize}
%
\begin{eqnarray}
  \hat{f}(p,\tau) = \int^{+\infty}_{-\infty} dx \int^{+\infty}_{-\infty} dt \delta(t-px-\tau)f(x,t),
                   \label{eq:radon}
\end{eqnarray}
%
\begin{itemize}
 \item $p$: slope 
 \item $\tau$ : intercept
 \item integration over $t$:
\end{itemize}
%
\begin{eqnarray}
  \hat{f}(p,\tau) = \int^{+\infty}_{-\infty} dx f(x,\tau+px).
                   \label{eq:slant}
\end{eqnarray}
%
\end{frame}
%
%==============================================
\begin{frame}{The Radon transform}
%==============================================
\sf
\begin{itemize}
\item Linear Radon transform or a slant stack
\item The line described by
\end{itemize}
%
\begin{eqnarray}
  t=p x+\tau,
\end{eqnarray}
% 
\begin{itemize}
  \item is mapped to $p,\tau$ in the Radon transformed domain.
\end{itemize}
\end{frame}
%
%==============================================
\begin{frame}{The Radon transform}
%==============================================
\begin{itemize}
 \item Fourier transform of equation \eqref{eq:slant} over $\tau$
\end{itemize}
%
\begin{eqnarray}
  \hat{f}(p,\omega) =   
      \frac{1}{2\pi}\int^{+\infty}_{-\infty}d\tau\,\int^{+\infty}_{-\infty} dx f(x,\tau+px)\exp(-i\omega\tau) 
                   \label{eq:slant-four}
\end{eqnarray}
%
\begin{itemize}
\item Integration variable $\tau$  changed to $u=\tau+px$ equation
\end{itemize}
\begin{eqnarray}
  \hat{f}(p,\omega) =
      \frac{1}{2\pi}\int^{+\infty}_{-\infty}du\,\int^{+\infty}_{-\infty} dx f(x,u)\exp(-i\omega u + \omega px)
                   \label{eq:slant-four2}
\end{eqnarray}
\begin{itemize}
 \item $k_x = \omega p$, one gets
\end{itemize}
\begin{eqnarray}
  \hat{f}(k_x/\omega,\omega) =
      \frac{1}{2\pi}\int^{+\infty}_{-\infty}du\,\int dx f(x,u)\exp(-i\omega u + k_x x).
                   \label{eq:slant-four2b}
\end{eqnarray}
\end{frame}
%==========================================
\begin{frame}{The Radon transform}
%==========================================
\sf
\begin{itemize}
  \item Inverse transform
\end{itemize}
%
\begin{eqnarray}
  f(x,t) = -\frac{1}{2\pi^2}\int dp\,\int d\tau\frac{\partial_{\tau}\hat{f}(p,\tau-px)} 
                                                   {\tau-t}.         
\end{eqnarray}
%
\end{frame}
%
%============================================================
\begin{frame}{The Radon transform of travel-time hyperbolas}
%=============================================================
The traveltime of a single primary reflection in a CMP-gather is
%
\begin{eqnarray}
t^2 = t^2_0 + x^2/c^2 . 
    \label{eq:5-hyp}
\end{eqnarray}
%
$c$ is the traveltime, and $t_0$ is the zero-offset traveltime.
The slownes $p$ is defined as
%
\begin{eqnarray}
  p = \frac{dt}{dx}, 
    \label{eq:5-pdef}
\end{eqnarray}
%
\end{frame}
%============================================================
\begin{frame}{The Radon transform of travel-time hyperbolas}
%============================================================
which gives by using equation \eqref{eq:5-hyp}
%
\begin{eqnarray}
 p = \frac{x}{tc^2}.
    \label{eq:5-p}
\end{eqnarray}
%
or
%
\begin{eqnarray}
 t = \frac{x}{pc^2}.
    \label{eq:5-t}
\end{eqnarray}
%
Inserting equation \eqref{eq:5-t} into equation 
\eqref{eq:5-hyp} gives
%
\begin{eqnarray}
x=\frac{pt_0 c^2}{\sqrt{1-p^2c^2}}.
    \label{eq:5-x}
\end{eqnarray}
Also inserting equation \eqref{eq:5-x} into \eqref{eq:5-t} gives
\begin{eqnarray}
  t=\frac{t_0}{\sqrt{1-p^2c^2}}.
    \label{eq:5-ta}
\end{eqnarray}
\end{frame}
%
%============================================================
\begin{frame}{The Radon transform of travel-time hyperbolas}
%============================================================
Using equations \eqref{eq:5-ta} and \eqref{eq:5-x} we get
\begin{eqnarray}
 \tau=t-px = t_0\sqrt{(1-p^2c^2)}.
    \label{eq:5-tau}
\end{eqnarray}
The last equation automatically gives
\begin{eqnarray}
 {\left(\frac{\tau}{c t_0}\right)}^2 + p^2 = \frac{1}{c^2},
    \label{eq:5-tau2}
\end{eqnarray}
which shows that the two parameters $\tau$ and $p$ lies on an ellipse.
An hyperbolic traveltime curve in the $x,t$ space is the transformed to an ellipse in
the $\tau-p$ space.
\end{frame}
%
%============================================================
\begin{frame}
%============================================================
\begin{figure}
\includegraphics[width=0.75\textwidth]{Fig/fig-5-shot.pdf}
%\plot{fig-5-shot}{width=0.75\textwidth}{CMP with two events with velocity equal to 2000 m/s and 2500 m/s.}
\caption{CMP with two events with velocity equal to 2000 m/s and 2500 m/s.}
\label{fig-5-shot}
\end{figure}
\end{frame}
%============================================================
\begin{frame}
%============================================================
\begin{figure}
\includegraphics[width=0.75\textwidth]{Fig/fig-5-taup.pdf}
%\plot{fig-5-taup}{width=0.75\textwidth}
\caption{$\tau-p$ transform of the CMP-gather shown in figure \protect{\ref{fig-5-shot}}}
\end{figure}
\end{frame}
%
%
%============================================================
\begin{frame}{Radon Multiple removal}
%============================================================
%
\begin{itemize}
  \item Raypaths with multiple reflections
  \item Multiples are unwanted
  \item Multiples are usually removed
\end{itemize}
%============================================================
\end{frame}
%
\begin{frame}{Principle of Radon demultiple}
Traveltime-distance curve for a primary
reflection from the bottom of a layer with constant velocity
%
\begin{eqnarray}
    t_p=\sqrt{\tau^2_0 + \frac{4h^2}{c^2}},
\end{eqnarray}
%
and the first multiple reflection from the same reflector
%
\begin{eqnarray}
    t_m=\sqrt{{(2\tau_0)}^2 + \frac{ 4{(2h)}^2}{c^2}}.
\end{eqnarray}
%
\end{frame}
%
%============================================================
\begin{frame}{Principle of Radon demultiple}
%============================================================
Consider also a primary reflection arriving  from some deeper
reflector at the same time 
%
\begin{eqnarray}
    t_p=\sqrt{{(2\tau_0)}^2 + \frac{ 4{(2h)}^2}{c_{rms}^2}}.
\end{eqnarray}
%
Although arriving at the same time as the multiple reflection, the curvature of
this event is different, because the velocity is $c_{rms} > c$.
\end{frame}
\begin{frame}{Principle of Radon demultiple}
%
\begin{figure}
\includegraphics[width=0.6\textwidth]{Fig/fig-6-cdp.pdf}
%\plot{fig-6-cdp}{width=0.6\textwidth}
\caption{Cmp with primary reflection and multiple reflection interfering.}
\label{fig-6-cdp}
\end{figure}
\end{frame}
%
%============================================================
\begin{frame}{Principle of Radon demultiple}
%============================================================
%
\begin{figure}
\includegraphics[width=0.6\textwidth]{Fig/fig-6-taup.pdf}
%\plot{fig-6-taup}{width=0.6\textwidth}
\caption{Cmp with primary reflection and multiple reflection from \protect{\ref{fig-6-cdp}} in the tau-p domain.
         The apparent velocity for the multiple reflection is lower than the primary reflection, 
         making separation of the two events possible.}
\end{figure}
%
\end{frame}
%
%============================================================
\begin{frame}{Principle of Radon demultiple}
%============================================================
\begin{figure}
\includegraphics[width=0.6\textwidth]{Fig/fig-6-nmo.pdf}
%\plot{fig-6-nmo}{width=0.6\textwidth}
\caption{Cmp with primary reflection and multiple reflection after nmo-correction
         with a velocity higher than the multiple velocity but lower than the primary velocity.}
 \label{fig-6-nmo}
\end{figure}
\end{frame}
%============================================================
\begin{frame}{Principle of Radon demultiple}
%============================================================
\begin{figure}
\includegraphics[width=0.6\textwidth]{Fig/fig-6-taup-nmo.pdf}
%\plot{fig-6-taup-nmo}{width=0.6\textwidth}
\caption{Cmp with primary reflection and multiple reflection  from figure \protect{\ref{fig-6-nmo}}
         after a $tau-p$-p transform. Only the multiple reflection is now visible, the primary
         appears for small negative p-values (not plotted)}.
\end{figure}
\end{frame}
%============================================================
\begin{frame}{Principle of Radon demultiple}
%============================================================
\begin{itemize}
  \item Straightforward mute and inverse transform is not possible
  \item Inverse radon gives too many artefacts
\end{itemize}
\end{frame}
%============================================================
\begin{frame}{Parabolic Radon}
%============================================================
%
\begin{eqnarray}
  \hat{f}(p,\tau) = \int^{+\infty}_{-\infty} dx \int^{+\infty}_{-\infty} dt \delta(t-px^2-\tau)f(x,t).
                   \label{eq:6-radon-para}
\end{eqnarray}
%

\begin{itemize}
 \item Transform maps an event in the x-t domain described by a parabola into a point
 \item Usefull for nmo-corrected data
\end{itemize}
\end{frame}
%============================================================
\begin{frame}{Parabolic Radon}
%============================================================
In general the inverse
transform can be considered to be of the general form
%
\begin{eqnarray}
  f(x,t) = \int^{+\infty}_{-\infty} dp \hat{f}(\tau=t-px^2,p),
                   \label{eq:6-para}
\end{eqnarray}
%
or in discrete form
%
\begin{eqnarray}
  f(x_k,t) = \sum_{l=0}^N \hat{f}(\tau=t- p_l x^2_k,p_l),
                   \label{eq:6-para-disc}
\end{eqnarray}
%
where $p_l=l\Delta p$ and $x_k = k\Delta x$. 
\end{frame}
%
%============================================================
\begin{frame}{Parabolic Radon}
%============================================================
We now want to perform a Fourier transform over the time variable
%
\begin{eqnarray}
  f(x_k,\omega) = \frac{1}{2\pi}\int^{+\infty}_{-\infty}dt \sum_{l=0}^N \hat{f}(\tau=t- p_l x^2_k,p_l)\exp(-i\omega t),
\end{eqnarray}
which becomes after substitution of variable $u=t-p_l x^2_l$
%
%
\begin{eqnarray}
  f(x_k,\omega) = \frac{1}{2\pi}\int^{+\infty}_{-\infty}du \sum_{l=0}^N \hat{f}(\tau=u,p_l)\exp[-i\omega (u+p_l x_k^2)].
\end{eqnarray}
%
The last equation is then
%
\begin{eqnarray}
  f(x_k,\omega) = \sum_{l=0}^N \hat{F}(\omega, p_l)\exp[-i\omega p_l x^2_k].
                   \label{eq:6-para-inv}
\end{eqnarray}
%
\end{frame}
%============================================================
\begin{frame}{Parabolic radon}
%============================================================
To compute the forward transform
we consider the $\hat{F}(\omega,p_l)$ as unknowns, and solve the linear system of
equations given in \eqref{eq:6-para-inv}. 
This can easily be done by writing equation \eqref{eq:6-para-inv} as a matrix equation
and using the least-squares method.
%
\begin{eqnarray}
  \mathbf{f}(\omega) = \mathbf{L}\hat{\mathbf{F}}(\omega)
                   \label{eq:6-para-inv2}
\end{eqnarray}
%
where $\mathbf{f}$ and $hat{\mathbf{F}}$ are vectors with elements $f_k=f(x_k,\omega)$ 
and $\hat{F}_k=\hat{F}(x_k)$. 
The matrix $\mathbf{L}$ have elements $L_{lk}=\exp[-i\omega p_l x^2_k$.
\end{frame}
%============================================================
\begin{frame}{Parabolic radon}
%============================================================
After muting of $\hat{\mathbf{F}}$ to remove multiples we solve the equation
\begin{eqnarray}
  \hat{\mathbf{F}}(\omega) = \mathbf{L}^{-1}\mathbf{f}(\omega)
                   \label{eq:6-para-inv3}
\end{eqnarray}
with respect to $\mathbf{f}$ to compute the inverse transform.
\end{frame}
%============================================================
\begin{frame}{Parabolic Radon}
%============================================================
%
\begin{figure}
%\plot{fig-6-data}{width=0.6\textwidth}
\includegraphics[width=0.6\textwidth]{Fig/fig-6-data.pdf} 
\caption{Input data with primary and multiple reflections. A normal moveout correction
         has been applied.} 
\label{fig-6-data}
\end{figure}
\end{frame}
%
%============================================================
\begin{frame}{Parabolic Radon}
%============================================================
\begin{figure}
%\plot{fig-6-radon}{width=0.6\textwidth}
\includegraphics[width=0.6\textwidth]{Fig/fig-6-radon.pdf} 
\caption{The parabolic radon transform of the data shown in figure \protect{\ref{fig-6-data}}.}
\end{figure}
\end{frame}
%============================================================
\begin{frame}{Parabolic Radon}
%============================================================
\begin{figure}
%\plot{fig-6-prim}{width=0.6\textwidth}
\includegraphics[width=0.6\textwidth]{Fig/fig-6-prim.pdf} 
\caption{Primary reflections estimated from the input data in figure \protect{\ref{fig-6-data}}
         using the parabolic radon transform.}
\end{figure}
\end{frame}
%
\end{document}
