\documentclass[xcolor=dvipsnames,notes]{beamer}
\usecolortheme[named=Brown]{structure}
\usetheme{default}
\setbeamertemplate{navigation symbols}{} 
\usepackage{tikz}
\usetikzlibrary{arrows,decorations.pathmorphing,backgrounds,positioning,fit}
\usetikzlibrary{datavisualization.formats.functions}
\usetikzlibrary{shapes}
\include{macro}
\usepackage{epsfig}
\usepackage{natbib}
\usepackage{graphicx}
\usepackage{multimedia}
\begin{document}
%\setbeamercolor{titlelike}{fg=gray,bg=white}
%\setbeamercolor{itemize item}{fg=gray,bg=white}
%\setbeamercolor{enumerate item}{fg=gray,bg=white}
%\setbeamercolor{block title}{fg=black,bg=white}
%==============================================
\title{TPG4190 Seismic data acquisition and processing \\
               Lecture 2: Receivers}
\author{B. Arntsen}
\institute[NTNU]{
  NTNU\\
  Department of Geoscience and petroleum \\
  \texttt{borge.arntsen@ntnu.no}
}
\date{Trondheim fall 2022}
\begin{frame}
 \titlepage
\end{frame}
%==============================================
%-----------------------------------------
\begin{frame}{Overview}
%-----------------------------------------
\begin{enumerate}
  \item Receiver system (sec. 2.8) 
\end{enumerate}
\end{frame}
%------------------------------------------------
\begin{frame}{Receiver system}
%------------------------------------------------
\begin{figure}
  \includegraphics[width=0.8\textwidth]{Fig/seismic-ship.jpg}
  \caption{Marine seismic setup}
  \label{fig:ship}
\end{figure}
\begin{itemize}
\item Each channel is a group of hydrophones
      (10-20) connected together
\item Distance between each group is typical 12.5m or 6.25m
\item Hydrophones are pizzo-electric devices sensitive to pressure
\end{itemize}
\end{frame}
%------------------------------------------------
\begin{frame}{Receiver system}
%------------------------------------------------
A single group of hydrophones are summed together as
\begin{eqnarray}
\sum_{j=0}^{N-1} a(x_j,),
                     \label{eq:array}
\end{eqnarray}
where $x_j = i\Delta x$
and $\Delta x$ is the distance between hydrophones and $N$
is the number, while $a(x_j)$ is the amplitude measured 
at position $x_j$.
Formally, this is the same as
\begin{eqnarray}
 g(x)= \sum_{j=0}^{N-1}\int_{0}^{x_j} dx\, \delta(x_j-x)a(x)
                     \label{eq:array2}
\end{eqnarray}
The expression inside the sum can be recognized as the convolution of $a(x)$
and $\delta(x_j-x)$. The Fourier transform of the convolution of two functions
is equal to the product of the Fourier transform of each function. Hence
we have
\begin{eqnarray}
G(k) = A(k)\sum_{j=0}^{N-1}\Delta(k,x_j)
\end{eqnarray}
\end{frame}
%---------------------------------------------------------------------
\begin{frame}{Receiver system}
%---------------------------------------------------------------------
where $\Delta$ is the Fourier transform of the delta function
\begin{eqnarray}
\Delta(k) = \exp(ikx_j) = \exp(ik j\Delta x)
\end{eqnarray}
Hence we have for the Fourier transform of the sum of the hydrophones
\begin{eqnarray}
H(k)=\sum_{j=0}^{N-1}\exp(ik j\Delta x).
\end{eqnarray}
This is a geometric series and the sum is equal to
%
\begin{eqnarray}
\sum_{j=0}^{N-1}\exp(ik j\Delta x) = \frac{1-\exp(ik N \Delta x)}
                                         {1-\exp(ik \Delta x)}.
\end{eqnarray}
%
The right hand side is rewritten as
%
\begin{eqnarray}
             \frac{\exp(ik  N\Delta x/2)
                   \left[\exp(-ikN\Delta x/2)
                  -\exp(ik N \Delta x/2)\right]}
                  {\exp(ik\Delta x/2)
                   \left[\exp(-ik\Delta x/2)-\exp(ik \Delta x/2)\right]}.
\end{eqnarray}
\end{frame}
%
%---------------------------------------------------------------------
\begin{frame}{Receiver system}
%---------------------------------------------------------------------
which is recognized as:
\begin{eqnarray}
H(k)  = 
                 \frac{\sin(kN\Delta x/2)}{\sin(k\Delta x/2)}
\end{eqnarray}
\begin{figure}
  \includegraphics[width=0.8\textwidth]{Fig/array-spec.png}
  \caption{Receiver group array spectrum}
  \label{fig:array}
\end{figure}
\end{frame}
%---------------------------------------------------------------------
\begin{frame}{Receiver system}
%---------------------------------------------------------------------
\begin{itemize}
\item Dynamic range of seismic data: 1$\mu$bar - 1 bar 
\item Seismic data often low-pass filtered at 125Hz
\item Old data often High-pass filtered at 3-5Hz
\item Temporal aliasing not a problem
\item Spatial aliasing almost always problematic
\end{itemize}
\end{frame}
%---------------------------------------------------------------------
\begin{frame}{Spatial aliasing}
%---------------------------------------------------------------------
\begin{itemize}
\item Spatial aliasing: Spacing between receivers is too large
                  to capture rapid variations of the data
\end{itemize}

\begin{eqnarray}
   \Delta r < \frac{v}{2f\sin(\theta)}.
\end{eqnarray}
where

\begin{itemize}
\item $\Delta r$ : Spacing between receiver (groups)
\item $v$        : Velocity at the receiver 
\item $\theta$   : Angle between vertical and direction of wave
\end{itemize}
 
\end{frame}
%---------------------------------------------------------------------
\begin{frame}{Receiver system}
%---------------------------------------------------------------------
\begin{figure}
  \includegraphics[width=0.8\textwidth]{Fig/birds.png}
  \caption{Streamer ambient noise}
  \label{fig:birds}
\end{figure}
\end{frame}
%---------------------------------------------------------------------
\begin{frame}{Receiver system}
%---------------------------------------------------------------------
\begin{figure}
  \includegraphics[width=0.8\textwidth]{Fig/noise1.png}
  \caption{Noise record for two different sources}
  \label{fig:noise1}
\end{figure}
\begin{eqnarray}
  P_{db}  =  20 \log\left( \frac{P}{P_{ref}}\right)
\end{eqnarray}
\end{frame}
%---------------------------------------------------------------------
\begin{frame}{Receiver system}
%---------------------------------------------------------------------
\begin{figure}
  \includegraphics[width=0.8\textwidth]{Fig/noise2.png}
  \caption{Noise record for single shot}
  \label{fig:noise1}
\end{figure}
\end{frame}
\end{document}

